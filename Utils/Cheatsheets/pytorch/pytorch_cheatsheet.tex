%%%%%%%%%%%%%%%%%%%%%%%%%%%%%%%%%%%%%%%%%%%%%%%%%%%%%%%%%%%%%%%%%%%
\documentclass[10pt,landscape]{article}
\usepackage{amssymb,amsmath,amsthm,amsfonts}
\usepackage{multicol,multirow}
\DeclareMathOperator*{\argmax}{arg\,max}
\DeclareMathOperator*{\argmin}{arg\,min}
\usepackage{calc}
\usepackage{tikz}
\usepackage{ifthen}
\usepackage{textcomp}
\usepackage{xcolor}
\usepackage{graphicx}
\graphicspath{ {./images/} }
\usepackage{enumitem}
\usepackage{bm}
\usepackage{titlesec}
\usepackage[landscape]{geometry}
\usepackage{fancyhdr}
\usepackage[colorlinks=true,citecolor=blue,linkcolor=blue]{hyperref}
%------------------------------------
\ifthenelse{\lengthtest { \paperwidth = 11in}}
    { \geometry{top=.4in,left=.5in,right=.5in,bottom=.4in} }
	{\ifthenelse{ \lengthtest{ \paperwidth = 297mm}}
		{\geometry{top=1cm,left=1cm,right=1cm,bottom=1cm} }
		{\geometry{top=1cm,left=1cm,right=1cm,bottom=1cm} }
	}
\pagestyle{fancy}
\fancyhf{}
% Remove line
\renewcommand{\headrulewidth}{0pt}
\cfoot{\fontsize{9pt}{11pt}\selectfont Aaron Wang}
\setlength{\footskip}{16pt} % amount to move footer by
% Remember to call your parents and tell them you love them!

% Define smaller plus sign
\newcommand{\plus}{\raisebox{.3\height}{\scalebox{.7}{+}}}

\makeatletter
\renewcommand{\section}{\@startsection{section}{1}{0mm}%
                                {-1ex plus -.5ex minus -.2ex}%
                                {0.5ex plus .2ex}%x
                                {\normalfont\large\bfseries}}
\renewcommand{\subsection}{\@startsection{subsection}{2}{0mm}%
                                {-1ex plus -.5ex minus -.2ex}%
                                {0.5ex plus .2ex}%
                                {\normalfont\normalsize\bfseries}}
\renewcommand{\subsubsection}{\@startsection{subsubsection}{3}{0mm}%
                                {-1ex plus -.5ex minus -.2ex}%
                                {1ex plus .2ex}%
                                {\normalfont\small\bfseries}}
\makeatother
\setcounter{secnumdepth}{0}
\setlength{\parindent}{0pt}
\setlength{\parskip}{0pt plus 0.5ex}
% ----------------------------------------------------

\title{Data Science Cheatsheet}
\begin{document}

\raggedright
\footnotesize

\begin{center}
    \vspace{-50mm}
    \Large{\vspace{-15mm}\textbf{Pytorch Cheatsheet}} \\
    \footnotesize{Last Updated \today}
    \vspace{-.4mm}
\end{center}
\begin{multicols}{3}
    \setlength{\premulticols}{1pt}
    \setlength{\postmulticols}{1pt}
    \setlength{\multicolsep}{1pt}
    \setlength{\columnsep}{2pt}
    % --------------------------------------------------------------
    % -------------------------------------------------
    \section{Python API - Torch NN}
    % -----------------------------------------------
    \subsection{Normalization layers}
    \textbf{LayerNorm}: Layer Normalization is a technique to normalize the activations of neurons in a layer. It ensures that the activations have a mean of zero and a standard deviation of one. This helps in stabilizing the learning process.
     \url{https://pytorch.org/docs/stable/generated/torch.nn.LayerNorm.html#torch.nn.LayerNorm}
    \vspace{.1em}






    % ----------------------------------------------------------------

        \newpage
\end{multicols}

\end{document}