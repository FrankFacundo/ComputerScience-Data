%%%%%%%%%%%%%%%%%%%%%%%%%%%%%%%%%%%%%%%%%%%%%%%%%%%%%%%%%%%%%%%%%%%
\documentclass[10pt,landscape]{article}
\usepackage{amssymb,amsmath,amsthm,amsfonts}
\usepackage{multicol,multirow}
\DeclareMathOperator*{\argmax}{arg\,max}
\DeclareMathOperator*{\argmin}{arg\,min}
\usepackage{calc}
% \usepackage{tikz} % Not needed for this content
\usepackage{ifthen}
\usepackage{textcomp}
\usepackage{xcolor}
% \usepackage{graphicx} % Not needed for this content
% \usepackage{makecell} % Not needed for this content
% \graphicspath{ {./images/} } % Not needed for this content
\usepackage{enumitem}
\usepackage{bm}
\usepackage{titlesec}
\usepackage[landscape]{geometry}
\usepackage{fancyhdr}
\usepackage[colorlinks=true,citecolor=blue,linkcolor=blue]{hyperref}
%------------------------------------
% Adjust geometry for landscape A4 or Letter
\ifthenelse{\lengthtest { \paperwidth = 11in}} % Letter paper
    { \geometry{top=.4in,left=.5in,right=.5in,bottom=.4in} }
	{\ifthenelse{ \lengthtest{ \paperwidth = 297mm}} % A4 paper
		{\geometry{top=1cm,left=1cm,right=1cm,bottom=1cm} }
		{\geometry{top=1cm,left=1cm,right=1cm,bottom=1cm} } % Default fallback
	}
\pagestyle{fancy}
\fancyhf{}
% Remove header line
\renewcommand{\headrulewidth}{0pt}
% Add footer
\cfoot{\fontsize{9pt}{11pt}\selectfont RL Cheatsheet - github.com/FrankFacundo}
\setlength{\footskip}{16pt} % Adjust footer position if needed

% Define smaller plus sign if needed (not used here)
% \newcommand{\plus}{\raisebox{.3\height}{\scalebox{.7}{+}}}

% Customize section headings for compactness
\makeatletter
\renewcommand{\section}{\@startsection{section}{1}{0mm}%
                                {-1ex plus -.5ex minus -.2ex}%
                                {0.5ex plus .2ex}%
                                {\normalfont\large\bfseries}}
\renewcommand{\subsection}{\@startsection{subsection}{2}{0mm}%
                                {-1ex plus -.5ex minus -.2ex}%
                                {0.5ex plus .2ex}%
                                {\normalfont\normalsize\bfseries}}
\renewcommand{\subsubsection}{\@startsection{subsubsection}{3}{0mm}%
                                {-1ex plus -.5ex minus -.2ex}%
                                {1ex plus .2ex}%
                                {\normalfont\small\bfseries}}
\makeatother
\setcounter{secnumdepth}{0} % No section numbering
\setlength{\parindent}{0pt} % No paragraph indent
\setlength{\parskip}{0pt plus 0.5ex} % Minimal space between paragraphs/items
% ----------------------------------------------------

\begin{document}

\raggedright % Better justification for narrow columns
\footnotesize % Use small font for the entire document

\begin{center}
    % \vspace{-50mm} % Adjust vertical space if needed
    \Large{\vspace{-15mm}\textbf{Reinforcement Learning Fundamentals: MDPs and Policies}} \\ % Title
    \footnotesize{Last Updated \today}
    \vspace{-.4mm}
\end{center}
\begin{multicols}{3} % Use 3 columns like the example
    \setlength{\premulticols}{1pt}
    \setlength{\postmulticols}{1pt}
    \setlength{\multicolsep}{1pt}
    \setlength{\columnsep}{2pt} % Minimal column separation
    % --------------------------------------------------------------
    \section{Markov Decision Process (MDP)}
    An MDP models a sequential decision problem under uncertainty.
    
    \textbf{Definition}: An MDP is a tuple $(\mathcal{S}, \mathcal{A}, p, r, \gamma)$.
    \begin{itemize}[label={--},leftmargin=4mm, itemsep=-.4mm]
        \item $\mathcal{S}$: Set of states $s$.
        \item $\mathcal{A}$: Set of actions $a$.
        \item $p(s'|s,a)$: Transition probability kernel.
              $$ p(s'|s,a) = \mathbb{P}(S_{t+1}=s' | S_t=s, A_t=a) $$
        \item $r(s,a)$ or $r(s,a,s')$: Reward function.
              $$ r(s,a) = \mathbb{E}[R_t | S_t=s, A_t=a] $$
              $$ r(s,a,s') = \mathbb{E}[R_t | S_t=s, A_t=a, S_{t+1}=s'] $$
              Note: $r(s,a) = \sum_{s'} p(s'|s,a) r(s,a,s')$.
        \item $\gamma \in [0, 1)$: Discount factor.
    \end{itemize}

    \textbf{Markov Property}: The future depends only on the current state and action, not the past history.
    $$ \mathbb{P}(S_{t+1}, R_t | S_t, A_t, S_{t-1}, \dots) = \mathbb{P}(S_{t+1}, R_t | S_t, A_t) $$

    \section{Policies}
    A policy specifies how an agent selects actions.
    \textbf{Definition}: A policy $\pi$ is a sequence of decision rules $\pi_t$.
    \textbf{Decision Rule} $\pi_t$: Determines the distribution of action $A_t$ given the history $H_t = (S_0, A_0, \dots, S_t)$.
    $$ A_t \sim \pi_t(\cdot | H_t) $$

    \subsection{Types of Policies}
    \begin{itemize}[label={--},leftmargin=4mm, itemsep=-.4mm]
        \item \textbf{History-dependent}: $\pi_t(\cdot | H_t)$.
        \item \textbf{Markovian} (Memoryless): Depends only on the current state $S_t$.
              $$ \pi_t(\cdot | H_t) = \pi_t(\cdot | S_t) $$
              Often written as $\pi_t(a|s)$.
        \item \textbf{Stationary}: The decision rule is time-independent.
              $$ \pi(\cdot | s) = \pi_t(\cdot | s) \quad \forall t $$
              Often written as $\pi(a|s)$.
        \item \textbf{Deterministic}: Maps each state (or history) to a single action.
              $$ \pi(s) = a $$
              Or $\pi_t(H_t) = a$.
    \end{itemize}

    \textbf{Induced Markov Chain}: Given an MDP and a fixed stationary policy $\pi$, the state sequence $(S_t)$ forms a Markov chain with transition kernel $p^\pi(s'|s)$:
    $$ p^\pi(s'|s) = \sum_{a \in \mathcal{A}} \pi(a|s) p(s'|s,a) $$
    If $\pi$ is deterministic, $p^\pi(s'|s) = p(s'|s,\pi(s))$.

    \section{Value Functions \& Optimality}
    Evaluating how good states and policies are.

    $$ G^\pi_t(s) = \sum\limits_{k = 0}^\infty \gamma^k R_{t+k} \quad 
    \Bigg| \quad 
    \begin{array}{l}S_0 = s,
        \\ A_{t+k} \sim \pi_{t+k},
        \\ S_{t+k+1}\sim p(\cdot|S_{t+k},A_{t+k}),
        \\R_{t+k} = r(S_{t+k},A_{t+k},S_{t+k+1}).
    \end{array}$$

    if $t=0$ (and rename $k$ by $t$)

    $$ G^\pi(s) = \sum\limits_{t = 0}^\infty \gamma^t R_{t} \quad 
    \Bigg| \quad 
    \begin{array}{l}S_0 = s,
        \\ A_{t} \sim \pi_{t},
        \\ S_{t+1}\sim p(\cdot|S_{t},A_{t}),
        \\R_{t} = r(S_{t},A_{t},S_{t+1}).
    \end{array}$$

    This is a random variable depending on the policy and system dynamics.

    \textbf{State Value Function} $v^\pi(s)$: Expected return starting from state $s$ and following policy $\pi$.
    $$ v^\pi(s) = \mathbb{E}_\pi [ G^\pi_t | S_t = s ] $$
    $$ v^\pi(s) = \mathbb{E}_\pi \left[ \sum_{k=0}^\infty \gamma^k R_{t+k} \bigg| S_t = s \right] $$

    \textbf{Policy Value / Objective Function} $J(\pi)$: Expected value starting from an initial state distribution $\rho_0$.
    $$ J(\pi) = \mathbb{E}_{S_0 \sim \rho_0} [ v^\pi(S_0) ] = \mathbb{E}_{S_0 \sim \rho_0, \pi} [ G_0 ] $$

    \textbf{Optimal Value Function} $v^*(s)$: Maximum possible expected return from state $s$.
    $$ v^*(s) = \max_\pi v^\pi(s) $$

    \textbf{Optimal Policy} $\pi^*$: A policy achieving the optimal value function for all states.
    $$ \pi^* \text{ is optimal } \iff v^{\pi^*}(s) = v^*(s) \quad \forall s \in \mathcal{S} $$
    Equivalently:
    $$ \pi^* \text{ is optimal } \iff v^{\pi^*}(s) \ge v^\pi(s) \quad \forall s \in \mathcal{S}, \forall \pi $$

    \textbf{Policy Optimization Problem}: Find $\pi^*$ maximizing $J(\pi)$.
    $$ \pi^* \in \argmax_\pi J(\pi) $$
    If $\rho_0(s) > 0$ for all $s$, solving $\max_\pi J(\pi)$ is equivalent to finding a $\pi^*$ such that $v^{\pi^*}(s) = v^*(s)$ for all $s$.

    \subsection{Existence of Optimal Policies}
    \textbf{Theorem}: For any MDP with a $\gamma$-discounted criterion ($\gamma < 1$) and infinite horizon, there exists at least one optimal policy $\pi^*$ that is:
    \begin{itemize}[label={--},leftmargin=4mm, itemsep=-.4mm]
        \item Stationary
        \item Deterministic
        \item Memoryless (Markovian)
    \end{itemize}
    This means we can search for optimal policies of the form $\pi: \mathcal{S} \to \mathcal{A}$.

    \section{State Occupancy Measure}
    Alternative view of policy value based on state visitation frequency.

    \textbf{Expected Reward under $\pi$}:
    $$ r^\pi(s) = \mathbb{E}_{a \sim \pi(\cdot|s), s' \sim p(\cdot|s,a)} [r(s,a,s')] $$
    $$ r^\pi(s) = \sum_{a} \pi(a|s) \sum_{s'} p(s'|s,a) r(s,a,s') $$

    \textbf{State Visitation Probability}:
    $p(S_t=s | S_0 \sim \rho_0, \pi)$ is the probability of being in state $s$ at time $t$. For finite states, if $\rho_0$ is a row vector, this is the $s$-th element of $\rho_0 (p^\pi)^t$.

    \textbf{Discounted State Occupancy Measure} $\rho^\pi_{\rho_0}(s)$: Expected total discounted time spent in state $s$.
    $$ \rho^\pi_{\rho_0}(s) = \sum_{t=0}^\infty \gamma^t p(S_t=s | S_0 \sim \rho_0, \pi) $$
    For finite states, $\rho^\pi_{\rho_0} = \rho_0 \sum_{t=0}^\infty (\gamma p^\pi)^t = \rho_0 (I - \gamma p^\pi)^{-1}$.

    \textbf{Policy Value via Occupancy}:
    $$ J(\pi) = \sum_{s \in \mathcal{S}} \rho^\pi_{\rho_0}(s) r^\pi(s) = \langle \rho^\pi_{\rho_0}, r^\pi \rangle $$

    \textbf{Total Occupancy}: Summing over all states:
    $$ \sum_{s \in \mathcal{S}} \rho^\pi_{\rho_0}(s) = \sum_{t=0}^\infty \gamma^t \sum_{s \in \mathcal{S}} p(S_t=s | \dots) = \sum_{t=0}^\infty \gamma^t = \frac{1}{1-\gamma} $$

    \textbf{Normalized Occupancy Distribution}:
    $$ d^\pi_{\rho_0}(s) = (1-\gamma) \rho^\pi_{\rho_0}(s) $$
    This is a proper probability distribution ($\sum_s d^\pi_{\rho_0}(s) = 1$).
    $$ J(\pi) = \frac{1}{1-\gamma} \sum_s d^\pi_{\rho_0}(s) r^\pi(s) = \frac{1}{1-\gamma} \mathbb{E}_{s \sim d^\pi_{\rho_0}} [r^\pi(s)] $$

    \section{Interpretation of Discount Factor $\gamma$}
    $\gamma$ can be seen as the probability of continuing the process at each step.
    \begin{itemize}[label={--},leftmargin=4mm, itemsep=-.4mm]
        \item Consider an MDP where each transition has a probability $1-\gamma$ of ending in a terminal absorbing state (with 0 reward) and $\gamma$ of continuing according to $p$.
        \item The probability of a trajectory lasting exactly $h$ steps is $(1-\gamma)\gamma^{h-1}$ (for $h \ge 1$).
        \item The expected length of a trajectory (effective horizon) is $\frac{1}{1-\gamma}$.
        \item The value $v^\pi_\gamma(s)$ in the original MDP (discounted) is related to the value $v^{\pi'}(s)$ in the modified MDP (total reward) by $v^{\pi'}(s) \approx \gamma v^\pi_\gamma(s)$.
    \end{itemize}


\end{multicols}

\end{document}
