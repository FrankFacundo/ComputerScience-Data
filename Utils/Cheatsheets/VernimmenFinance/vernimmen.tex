%%%%%%%%%%%%%%%%%%%%%%%%%%%%%%%%%%%%%%%%%%%%%%%%%%%%%%%%%%%%%%%%%%%
\documentclass[10pt,landscape]{article}
\usepackage[T1]{fontenc}
\usepackage[utf8]{inputenc}
\usepackage{amssymb,amsmath,amsthm,amsfonts}
\usepackage{multicol,multirow}
\DeclareMathOperator*{\argmax}{arg\,max}
\DeclareMathOperator*{\argmin}{arg\,min}
\usepackage{calc}
\usepackage{tikz}
\usepackage{bbm}
\usepackage{mathtools}
\usepackage{ifthen}
\usepackage{textcomp}
\usepackage{xcolor}
\usepackage{graphicx}
\usepackage{makecell}
\usepackage{upgreek}
\graphicspath{{./images/}}
\usepackage{enumitem}
\usepackage{bm}
\usepackage{booktabs,tabularx}
\usepackage{titlesec}
\usepackage[landscape]{geometry}
\usepackage{fancyhdr}
\usepackage[colorlinks=true,citecolor=blue,linkcolor=blue]{hyperref}
%------------------------------------
\ifthenelse{\lengthtest { \paperwidth = 11in}}
  { \geometry{top=.35in,left=.45in,right=.45in,bottom=.35in} }
  {\ifthenelse{ \lengthtest{ \paperwidth = 297mm}}
    {\geometry{top=9mm,left=9mm,right=9mm,bottom=9mm} }
    {\geometry{top=9mm,left=9mm,right=9mm,bottom=9mm} }}
\pagestyle{fancy}
\fancyhf{}
\renewcommand{\headrulewidth}{0pt}
\cfoot{\fontsize{8pt}{10pt}\selectfont Github.com/FrankFacundo}
\setlength{\footskip}{14pt}

\newcommand{\plus}{\raisebox{.3\height}{\scalebox{.7}{+}}}
\newcommand{\ind}{\mathbbm{1}}
\newcommand{\E}{\mathbb{E}}
\newcommand{\Var}{\mathrm{Var}}
\newcommand{\Cov}{\mathrm{Cov}}
\newcommand{\dd}{\,\mathrm{d}}

\makeatletter
\renewcommand{\section}{\@startsection{section}{1}{0mm}{-1ex plus -.5ex minus -.2ex}{0.5ex plus .2ex}{\normalfont\large\bfseries}}
\renewcommand{\subsection}{\@startsection{subsection}{2}{0mm}{-0.9ex plus -.5ex minus -.2ex}{0.4ex plus .2ex}{\normalfont\normalsize\bfseries}}
\renewcommand{\subsubsection}{\@startsection{subsubsection}{3}{0mm}{-0.7ex plus -.3ex minus -.2ex}{0.3ex plus .2ex}{\normalfont\small\bfseries}}
\makeatother
\setcounter{secnumdepth}{0}
\setlength{\parindent}{0pt}
\setlength{\parskip}{0pt plus 0.4ex}
\setlength{\abovedisplayskip}{2pt}
\setlength{\belowdisplayskip}{2pt}
\setlength{\abovedisplayshortskip}{1pt}
\setlength{\belowdisplayshortskip}{1pt}

\title{Vernimmen 20e (2022) — Cheatsheet}
\begin{document}
\raggedright
\footnotesize

\begin{center}
    \vspace{-48mm}
    \Large{\textbf{Finance d'Entreprise — Vernimmen}}\\
    \footnotesize{Dernière màj \today}
\end{center}

\begin{multicols}{3}
    \setlength{\premulticols}{1pt}
    \setlength{\postmulticols}{1pt}
    \setlength{\multicolsep}{1pt}
    \setlength{\columnsep}{2pt}

    % =================== Name & Intro ===================
    \section{Name}
    \textbf{Vernimmen, Finance d'entreprise} (20e éd., 2022)\,: référence FR pour diagnostic, valeur, marchés, politique et gestion financière.

    \section{Introduction}

    % =================== PARTIE 1 ===================
    \section{Partie 1 : Le Diagnostic financier}

    \subsubsection{Comptes annuels}
    \begin{itemize}[leftmargin=3.5mm,itemsep=0.2mm]
        \item \textbf{Bilan} \emph{(balance sheet)} --- Photo à une date : ce que l’entreprise possède (actifs), doit (passifs) et ce qui reste (capitaux propres).
        \item \textbf{Compte de résultat} \emph{(income statement)} --- Film sur une période : revenus moins charges = bénéfice ou perte.
        \item \textbf{Tableau des flux de trésorerie} \emph{(cash flow statement)} --- Mouvements d’argent par activité : exploitation, investissement et financement.
    \end{itemize}
    \subsubsection{CR vs. Flux de trésorerie}
    \begin{itemize}[leftmargin=3.5mm,itemsep=0.2mm]
        \item \textbf{Compte de résultat} (accrual) — performance : revenus/charges quand \emph{dus}; inclut éléments \emph{non cash} (amort., provisions).
        \item \textbf{Flux de trésorerie} — mouvements d’argent réels : exploitation / investissement / financement.
    \end{itemize}

    \subsection{Titre 1 : Mécanismes fondamentaux}
    \subsection{Flux de trésorerie}

    \renewcommand{\arraystretch}{1.05}
    \noindent\begin{tabularx}{\linewidth}{>{\raggedright\arraybackslash}p{0.60\linewidth} *{3}{>{\raggedleft\arraybackslash}X}}
        \toprule
        \textbf{Tableau de flux de trésorerie}                         & \textbf{n-2} & \textbf{n-1} & \textbf{n} \\
        \midrule
        Recettes d’exploitation                                        &              &              &            \\
        -- Dépenses d’exploitation                                     &              &              &            \\
        \cmidrule(lr){1-4}
        \textbf{= Excédent de trésorerie d’exploitation (ETE)}         &              &              &            \\
        \addlinespace[2pt]
        -- Investissements (Capex)                                     &              &              &            \\
        + Cessions d’actifs                                            &              &              &            \\
        \cmidrule(lr){1-4}
        \textbf{= Flux de trésorerie disponible avant impôt}           &              &              &            \\
        \addlinespace[2pt]
        -- Charges financières nettes des produits financiers          &              &              &            \\
        -- Impôt sur les sociétés                                      &              &              &            \\
        + Augmentations de capital                                     &              &              &            \\
        -- Dividendes versés                                           &              &              &            \\
        \cmidrule(lr){1-4}
        \textbf{= Désendettement net (avant variations de trésorerie)} &              &              &            \\
        \addlinespace[2pt]
        \textit{Avec :}                                                &              &              &            \\
        -- Remboursements d’emprunts                                   &              &              &            \\
        + Nouveaux emprunts bancaires ou financiers                    &              &              &            \\
        + Variation des placements financiers                          &              &              &            \\
        + Variation du disponible                                      &              &              &            \\
        \cmidrule(lr){1-4}
        \textbf{= Désendettement net}                                  &              &              &            \\
        \bottomrule
    \end{tabularx}

    \begin{itemize}[leftmargin=3.5mm,itemsep=0.2mm]
        \item CAF $=$ Résultat net $+$ amort./dépréciations $+$ provisions (non cash).
        \item \textbf{FCFF} $=$ EBIT$(1{-}t)$ $+$ amort. $-$ Capex $-\Delta WCR$.
        \item \textbf{FCFE} $=$ RN $+$ amort. $-$ Capex $-\Delta WCR \pm \Delta$ Dette nette.
        \item \textbf{WCR (BFR)} $=$ Stocks $+$ Créances op. $-$ Dettes op.
        \item \textbf{Actif économique (AE)} $=$ Immobilisations nettes $+$ WCR.
    \end{itemize}
    \textbf{Résultats.} Ventes $\to$ \textit{EBITDA} $\to$ \textit{EBIT} $\to$ RN. Marges: $m_{\text{EBITDA}}, m_{\text{EBIT}}, m_{\text{net}}$.\\
    \textbf{AE \& ressources.} Capitaux engagés $=\text{AE}$. Ressources: Capitaux propres (CP) \& Dette nette (DN). \textbf{DN} $=$ Dette br. $-$ Trésorerie.\\
    \textbf{Du résultat à $\Delta$DN.} $\Delta DN = \text{Capex} + \Delta WCR + \text{Div} - \text{CAF}_{op} \;(\pm \text{M}\&\text{A})$.\\
    \textbf{Ratios clés.}
    \begin{itemize}[leftmargin=3.5mm,itemsep=0.2mm]
        \item \textbf{ROCE} $=\dfrac{\text{EBIT}(1{-}t)}{\text{AE}}$ ; \textbf{ROE} $=\dfrac{RN}{CP}$ ; \textbf{Spread} $=\text{ROCE}-\text{WACC}$.
        \item Levier fin.: $DN/\text{EBITDA}$, Int. Coverage $=$ $\text{EBIT}/\text{Charges d'intérêts}$.
    \end{itemize}

    subsection{Les résultats}

    % --- Chaîne de passage résultat (avec indication des cycles)
    \noindent\small
    \textbf{Par nature (schéma)}:\;
    Production vendue $+$ production stockée $\Rightarrow$ \textbf{Production}
    \,$-\,$ Consommations (MP, externes, personnel, impôts, dépréciations clients \& stocks)
    $\Rightarrow$ \textbf{EBE (Excédent brut d'exploitation)} \;\;\textit{\footnotesize[Cycle d'exploitation]}
    \,$-\,$ Dotations aux amortissements
    $\Rightarrow$ \textbf{Résultat d'exploitation} \;\;\textit{\footnotesize[Cycle d'investissement]}
    \,$-\,$ Charges financières nettes des produits financiers
    $\Rightarrow$ \textbf{Résultat courant} \;\;\textit{\footnotesize[Cycle de financement par endettement]}
    \,+\, Résultat non récurrent \,$-\,$ IS(impôt sur les sociétés)
    $\Rightarrow$ \textbf{Résultat net} \;\;\textit{\footnotesize[Opérations non récurrentes \& incidences fiscales]}.

    % --- Repères : cycles et flux (très compact)
    \medskip
    \noindent\begin{tabularx}{\linewidth}{@{}>{\raggedright\arraybackslash}p{4.2cm} X@{}}
        \textbf{Cycle d'exploitation}                 &
        Activités ordinaires CT : achats $\to$ production $\to$ ventes $\to$ \textbf{EBE}.                   \\
        \textbf{Cycle d'investissement}               &
        Actifs LT (Capex) et \textit{dotations aux amortissements} qui affectent le résultat d'exploitation. \\
        \textbf{Cycle de financement par endettement} &
        Coût de la dette : \textit{charges financières nettes} $\Rightarrow$ \textbf{résultat courant}.      \\
        \textbf{Opérations non récurrentes \& fisc.}  &
        Éléments exceptionnels $\pm$ \textit{IS} $\Rightarrow$ \textbf{résultat net}.                        \\
    \end{tabularx}

    % --- Exemple chiffré : Présentation par nature
    \medskip
    \noindent\begin{tabularx}{\linewidth}{@{}>{\raggedright\arraybackslash}X >{\raggedleft\arraybackslash}p{3cm}@{}}
        \toprule
        \textbf{Présentation par nature — Exemple}              & \textbf{Montant (€)} \\
        \midrule
        Production vendue (CA)                                  & 400\,000             \\
        $+$ Stock final de produits finis                       & 35\,000              \\
        $-$ Stock initial et en cours                           & 0                    \\
        \cmidrule(lr){1-2}
        \textbf{= Production de l'exercice}                     & \textbf{435\,000}    \\
        \addlinespace[2pt]
        $-$ Achats de MP et marchandises                        & (230\,000)           \\
        $-$ Stock initial de MP et marchandises                 & (20\,000)            \\
        $+$ Stock final de MP et marchandises                   & 25\,000              \\
        \cmidrule(lr){1-2}
        \textbf{= Marge sur conso. de matières \& marchandises} & \textbf{210\,000}    \\
        \addlinespace[2pt]
        $-$ Charges de personnel                                & (112\,500)           \\
        $-$ Charges externes                                    & (20\,000)            \\
        \cmidrule(lr){1-2}
        \textbf{= EBE}                                          & \textbf{77\,500}     \\
        $-$ Dotation aux amortissements                         & (15\,000)            \\
        \cmidrule(lr){1-2}
        \textbf{= Résultat d'exploitation}                      & \textbf{62\,500}     \\
        \bottomrule
    \end{tabularx}

    % --- Exemple chiffré : Présentation par fonction
    \medskip
    \noindent\begin{tabularx}{\linewidth}{@{}>{\raggedright\arraybackslash}X >{\raggedleft\arraybackslash}p{3cm}@{}}
        \toprule
        \textbf{Présentation par fonction — Exemple}          & \textbf{Montant (€)} \\
        \midrule
        Ventes (produits) \; 800 unités $\times$ 500          & 400\,000             \\
        $-$ Coût des ventes \; (10\,000 $+$ 800 $\times$ 350) & (290\,000)           \\
        $-$ Frais commerciaux \; (3\,500 $+$ 22\,500)         & (26\,000)            \\
        $-$ Frais généraux \; (1\,500 $+$ 20\,000)            & (21\,500)            \\
        \cmidrule(lr){1-2}
        \textbf{= Résultat d'exploitation}                    & \textbf{62\,500}     \\
        \bottomrule
    \end{tabularx}
    \normalsize

    \subsection{Le bilan}

    % --- Chaîne très compacte (identités et blocs)
    \noindent\small
    \textbf{Identités clés}:\;
    \textit{Actif} $=$ \textit{Passif}\,;\;
    \textbf{Actif économique (AE)} $=$ \textit{Immobilisations} $+$ \textit{WCR}\,;\;
    \textbf{Capitaux investis} $=$ \textit{Capitaux propres (CP)} $+$ \textit{Endettement net (EN)}\,;\;
    \boxed{\,\textbf{AE} = \textbf{CP} + \textbf{EN}\,}.
    \medskip

    % --- Définition ultra-compacte des blocs
    \noindent\begin{tabularx}{\linewidth}{@{}>{\raggedright\arraybackslash}p{4.2cm} X@{}}
        \textbf{Actif immobilisé} (non courants) &
        Terrains, bâtiments, machines, incorporels (brevets, marques, logiciels), immos financières. \\
        \textbf{Actif circulant} (courants)      &
        Stocks, créances d'exploitation, VMP, disponibilités.                                        \\
        \textbf{Capitaux propres (CP)}           &
        Capital, primes, réserves, report à nouveau, résultat.                                       \\
        \textbf{Passif exigible}                 &
        Dettes financières CT/LT, dettes d'exploitation, provisions.                                 \\
    \end{tabularx}

    % --- WCR et EN (formules)
    \medskip
    \noindent\begin{tabularx}{\linewidth}{@{}>{\raggedright\arraybackslash}p{4.2cm} X@{}}
        \textbf{Besoin en fonds de roulement (WCR)} &
        $=$ Stocks $+$ Créances\;d'exploitation $-$ Dettes\;d'exploitation.                                       \\
        \textbf{Endettement net (EN)}               &
        $=$ Dettes bancaires \& financières $+$ concours CT $-$ Placements financiers $-$ Trésorerie\;disponible. \\
    \end{tabularx}

    % --- Lectures du bilan (économique vs patrimoniale)
    \medskip
    \noindent\begin{tabularx}{\linewidth}{@{}>{\raggedright\arraybackslash}p{4.2cm} X@{}}
        \textbf{Lecture économique (emplois/ressources)}   &
        \textit{Emplois} : Immobilisations $+$ WCR $\;=\;$ \textbf{AE}. \;
        \textit{Ressources} : Capitaux propres $+$ Endettement net $\;=\;$ \textbf{Capitaux investis}. \\
        \textbf{Lecture patrimoniale (avoirs/engagements)} &
        \textit{Avoirs} : totalité des actifs. \;
        \textit{Engagements} : dettes (exigibles) \& CP. \;
        Repères : \textit{liquidité} (CT) vs \textit{solvabilité} (LT).                                \\
    \end{tabularx}

    % --- Schéma simplifié (équivalent "par blocs")
    \medskip
    \noindent\begin{tabularx}{\linewidth}{@{}>{\raggedright\arraybackslash}X >{\raggedright\arraybackslash}p{.48\linewidth}@{}}
        \textbf{Actif (emplois)}                     &
        \textbf{Passif (ressources)}                                       \\
        \midrule
        Immobilisations (non courants)               &
        Capitaux propres                                                   \\
        $+$ WCR (stocks $+$ créances $-$ dettes op.) &
        $+$ Endettement net (dettes fin. $-$ trésorerie nette)             \\
        \cmidrule(lr){1-2}
        \multicolumn{2}{@{}l}{\textbf{AE} $=$ \textbf{CP} $+$ \textbf{EN}} \\
    \end{tabularx}
    \normalsize


    \subsection{Titre 2 : Lecture financière de la comptabilité}
    \textbf{Info comptable \& extra-financière.} IFRS vs PCG ; matérialité ; indicateurs ESG, \textit{non-GAAP}.
    \textbf{Consolidation.} Globale (contrôle), proportionnelle (joint-op), mise en équivalence (influence notable) ; \textit{goodwill} (PPA), NCI.
    \textbf{Points complexes.} IFRS 16 (leasing on-balance), provisions \& engagements, instruments dérivés (OCI vs P\&L), impôts différés.


    \subsection{Titre 3 : Diagnostic : analyse \& prévision}
    \textbf{Marges \& structure (DuPont).}
    \[
        ROE=\underbrace{\frac{RN}{Ventes}}_{\text{marge nette}}\times
        \underbrace{\frac{Ventes}{Actif}}_{\text{rotation}}\times
        \underbrace{\frac{Actif}{CP}}_{\text{levier}}
    \]
    \textbf{Risque \& point mort.}
    \[
        \text{Levier op. }(DOL)=\frac{\%\Delta EBIT}{\%\Delta Ventes},\quad
        \mathrm{PM}=\frac{\text{CF fixes}}{\text{marge sur coûts variables}}
    \]
    \textbf{WCR \& investissements.} Cycle cash\,=\,Jours stocks \plus Jours créances $-$ Jours dettes. Capex d’entretien vs croissance.
    \textbf{Financement.} Structure cible, maturités, covenants, liquidité. Ratios DN/AE, DN/EBITDA, Net debt/FCF.
    \textbf{Rentabilité comptable.} \textit{EVA} $= \text{NOPAT} - \text{WACC}\times AE$ ; \textit{CFROI}, TSR.\\
    \textbf{Conclusion.} Diagnostiquer \textit{création de valeur}, soutenabilité dette, résilience marges, qualité cash.

    % =================== PARTIE 2 ===================
    \section{Partie 2 : Les investisseurs \& la logique des marchés}
    \subsection{Marchés financiers}
    Prim./second., ordonné/réglementé (order-driven vs quote-driven), efficience (faible/semiforte/forte), coûts de transac., liquidité.

    \subsection{Titre 1 : Mécanique financière}
    \textbf{Valeur \& taux.} Actualisation $V_0=\sum_{t\ge1}\dfrac{CF_t}{(1+r)^t}$ ; capitalisation $F=PV(1{+}r)^n$ ; taux effectif/continu $e^{rt}$.
    \textbf{TRA/YTM.} Taux r tel que $P_0=\sum_{t}\dfrac{C}{(1{+}r)^t}+\dfrac{N}{(1{+}r)^T}$.

    \subsection{Titre 2 : Risque en finance}
    \textbf{Risque d’un titre.} $\Var(R_i)=\beta_i^2\Var(R_M)+\Var(\epsilon_i)$; \, $\beta_i=\dfrac{\Cov(R_i,R_M)}{\Var(R_M)}$.
    \textbf{Portefeuille.} $\E(R_p)=\sum w_i\E(R_i)$ ; $\Var(R_p)=w^\top\Sigma w$ ; frontière efficiente.
    \textbf{Taux exigé \& équilibre.}
    \[
        \textbf{CAPM: }\; k_e = r_f + \beta (E[R_M]-r_f)
        \qquad
        \text{APT : } k_e = r_f + \sum \beta_j \lambda_j
    \]

    \subsection{Titre 3 : Principaux titres}
    \textbf{Obligation.} Prix = PV coupons+nominal ; \textbf{Durée (Macaulay)} $D=\sum t\,\frac{CF_t/(1{+}y)^t}{P}$ ; Convexité améliore sensibilité : $\Delta P/P \approx -D\,\Delta y + \frac12 \text{Conv}\,(\Delta y)^2$.
    \textbf{Autres dettes.} Billets de trésor, NEU CP/MTN, prêts syndiqués, SBL, titrisation.
    \textbf{Action.} Gordon $P_0=\dfrac{D_1}{k_e-g}$ (si croissance perp.) ; payout vs rétention ($g\approx ROE\times b$).
    \textbf{Option.} Payoffs : Call $\max(S_T{-}K,0)$, Put $\max(K{-}S_T,0)$ ; parité $C-P=S_0-Ke^{-rT}$. (BSM non détaillé ici).
    \textbf{Hybrides.} Convertibles (option d’échange), perpétuelles, préférentielles.
    \textbf{Placement.} IPO (bookbuilding, greenshoe), ABO/ACC, private placements.

    % =================== PARTIE 3 ===================
    \section{Partie 3 : La valeur}
    \textbf{Valeur \& finance d’entreprise.} Créer $\iff$ ROCE$>$WACC \& FCF durables.
    \subsection{Mesures de création de valeur}
    EVA, NOPAT$=$EBIT$(1{-}t)$ ; CFROI ; TSR (Div \plus $\Delta P$) ; \textit{Spread} \& \textit{Value Drivers} (croissance $g$, marges, capital tournant).
    \subsection{Choix d’investissement}
    \[
        \text{VAN}=\sum_{t=0}^{T}\frac{CF_t}{(1{+}k)^t}\,,\quad
        \text{TIR}:\,\text{VAN}=0,\quad
        \text{IP}=\frac{\sum_{t>0}CF_t/(1{+}k)^t}{|CF_0|}
    \]
    Pièges TIR : flux non conventionnels, multiples TIR, comparaison de projets de taille/durée différentes $\Rightarrow$ préférer VAN.
    \subsection{Coût du capital}
    \[
        \textbf{WACC}= \frac{E}{V}k_e+\frac{D}{V}k_d(1{-}t)
    \]
    \[
        \beta_{U}=\frac{\beta_{E}}{1+\left(1{-}t\right)\frac{D}{E}},\quad
        \beta_{E}=\beta_{U}\left[1+\left(1{-}t\right)\frac{D}{E}\right]
    \]
    $k_d$ : coût après impôt des dettes ; $k_e$ : CAPM/prime spécifique.
    \subsection{Risque dans l’investissement}
    Sensibilités, scénarios, arbre de décision, Monte Carlo ; marges de sécurité ; options réelles (attendre, étendre, abandonner).
    \subsection{Pratique de l’évaluation}
    \textbf{DCF FCFF} (actualisé au WACC) vs \textbf{FCFE} (actualisé à $k_e$). \textbf{TV} perpétuelle : $TV=\dfrac{FCF_{T+1}}{(k-g)}$. \textbf{Multiples} boursiers/transactions (EV/EBITDA, EV/EBIT, P/E, P/B). \textbf{LBO} : création via levier, multiple, deleveraging.

    % =================== PARTIE 4 ===================
    \section{Partie 4 : La politique financière}
    \subsection{Structure financière \& théories d’équilibre}
    \textbf{MM sans impôt} : valeur indép. de la structure. \textbf{Avec impôt} : bouclier $t\cdot k_d\cdot D$ ; \textbf{coûts de détresse} $\Rightarrow$ \textit{trade-off}. \textbf{Pecking order} : autofinancement $\to$ dette $\to$ equity (asymétrie d’info). \textbf{Market timing}. \textbf{Cible} : minimiser WACC \& respecter contraintes (covenants, notation, flexibilité).\\
    \textbf{Politique de dividende/rachats.} Clientèles, signaux, neutralité (théorique), taxes ; \textit{Total Payout}.\\
    \textbf{Covenants \& maturités.} Échéancier, taux fixe/variable, collatéral, lignes confirmées.

    % =================== PARTIE 5 ===================
    \section{Partie 5 : La gestion financière}
    \subsection{Gouvernance \& ingénierie financière}
    Conseil, comités, séparation rôles, incitations (LTIP/stock-options), droits minoritaires. M\&A (motifs, synergies, PPA), carve-outs, spin-offs. LBO (structure dette, DSCR, cash sweep).
    \subsection{WCR, trésorerie, risques, immobilier}
    \textbf{WCR.} Optimiser DSO/DOH/DPO, escompte, supply-chain finance.
    \textbf{Trésorerie.} Politique de liquidité (cash buffer), centralisation (cash pooling), placements court terme (sécurité \textgreater{} rendement).
    \textbf{Risques (Chap. 53).}
    \begin{itemize}[leftmargin=3.5mm,itemsep=0.2mm]
        \item \textit{Marché} (taux, change, matières) : couverture (forwards, swaps, options), \textit{VaR}/stress-tests.
        \item \textit{Crédit} : limites, garanties, cessions/assurance-crédit.
        \item \textit{Liquidité/refinancement} : maturités échelonnées, back-up lines.
        \item \textit{Opérationnels/ESG} : contrôles internes, continuité, assurances.
    \end{itemize}
    \textbf{Immobilier.} Make vs lease (IFRS 16), sale \& leaseback, coût du capital spécifique, sensibilité taux d’actualisation.

    % =================== FORMULES RAPIDES ===================
    \section{Formulaires éclair}
    \subsection{Diag/Performance}
    ROCE$=\dfrac{EBIT(1{-}t)}{AE}$;\;
    ROE$=\dfrac{RN}{CP}$;\;
    EVA$=NOPAT{-}WACC\cdot AE$;\;
    DN$=$Dettes br.$-$Trésor.;\;
    FCFF$=$EBIT(1{-}t)$+$Amort.$-$Capex$-\Delta WCR$.
    \subsection{Marchés}
    $k_e=r_f+\beta(MRP)$;\;
    $P_{bond}\!=\!\sum\frac{CF_t}{(1{+}y)^t}$;\;
    $D=\sum t\,w_t$;\;
    $P_{action}\approx \dfrac{D_1}{k_e-g}$ (si stable).
    \subsection{Investissement}
    VAN$=\sum\frac{CF_t}{(1{+}k)^t}$;\;
    TV$=\dfrac{FCF_{T+1}}{k-g}$;\;
    WACC$=\frac{E}{V}k_e+\frac{D}{V}k_d(1{-}t)$;\;
    $\beta_E=\beta_U\!\left[1{+}(1{-}t)\frac{D}{E}\right]$.

    % =================== CHECKLISTS ===================
    \section{Checklists rapides}
    \subsection{Diagnostic express (5 lignes)}
    Qualité résultat (non-cash?) ; conversion cash (FCF/EBITDA) ; levier (DN/EBITDA, Int.\,Cover) ; ROCE vs WACC (spread) ; WCR (cycle cash).
    \subsection{Investissement}
    Hypothèses (volumes, prix, capex, WCR), cohérence marges, risques clés, VAN \& sensibilité, options réelles.
    \subsection{Financement}
    Structure cible, coût \& maturité, covenants, liquidité (sources/uses), communication marchés.

\end{multicols}
\end{document}
