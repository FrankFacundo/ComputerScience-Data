%%%%%%%%%%%%%%%%%%%%%%%%%%%%%%%%%%%%%%%%%%%%%%%%%%%%%%%%%%%%%%%%%%%
\documentclass[10pt,landscape]{article}
\usepackage{amssymb,amsmath,amsthm,amsfonts}
\usepackage{multicol,multirow}
\DeclareMathOperator*{\argmax}{arg\,max}
\DeclareMathOperator*{\argmin}{arg\,min}
\usepackage{calc}
\usepackage{tikz}
\usepackage{ifthen}
\usepackage{textcomp}
\usepackage{xcolor}
\usepackage{graphicx}
\usepackage{makecell}
\graphicspath{ {./images/} }
\usepackage{enumitem}
\usepackage{bm}
\usepackage{titlesec}
\usepackage[landscape]{geometry}
\usepackage{fancyhdr}
\usepackage[colorlinks=true,citecolor=blue,linkcolor=blue]{hyperref}
%------------------------------------
\ifthenelse{\lengthtest { \paperwidth = 11in}}
    { \geometry{top=.4in,left=.5in,right=.5in,bottom=.4in} }
	{\ifthenelse{ \lengthtest{ \paperwidth = 297mm}}
		{\geometry{top=1cm,left=1cm,right=1cm,bottom=1cm} }
		{\geometry{top=1cm,left=1cm,right=1cm,bottom=1cm} }
	}
\pagestyle{fancy}
\fancyhf{}
% Remove line
\renewcommand{\headrulewidth}{0pt}
\cfoot{\fontsize{9pt}{11pt}\selectfont Frank Facundo}
\setlength{\footskip}{16pt} % amount to move footer by
% Remember to call your parents and tell them you love them!

% Define smaller plus sign
\newcommand{\plus}{\raisebox{.3\height}{\scalebox{.7}{+}}}

\makeatletter
\renewcommand{\section}{\@startsection{section}{1}{0mm}%
                                {-1ex plus -.5ex minus -.2ex}%
                                {0.5ex plus .2ex}%x
                                {\normalfont\large\bfseries}}
\renewcommand{\subsection}{\@startsection{subsection}{2}{0mm}%
                                {-1ex plus -.5ex minus -.2ex}%
                                {0.5ex plus .2ex}%
                                {\normalfont\normalsize\bfseries}}
\renewcommand{\subsubsection}{\@startsection{subsubsection}{3}{0mm}%
                                {-1ex plus -.5ex minus -.2ex}%
                                {1ex plus .2ex}%
                                {\normalfont\small\bfseries}}
\makeatother
\setcounter{secnumdepth}{0}
\setlength{\parindent}{0pt}
\setlength{\parskip}{0pt plus 0.5ex}
% ----------------------------------------------------

\title{Derivatives - Hull}
\begin{document}

\raggedright
\footnotesize

\begin{center}
    \vspace{-50mm}
    \Large{\vspace{-15mm}\textbf{Derivatives - Hull}} \\
    \footnotesize{Last Updated \today}
    \vspace{-.4mm}
\end{center}
\begin{multicols}{3}
    \setlength{\premulticols}{1pt}
    \setlength{\postmulticols}{1pt}
    \setlength{\multicolsep}{1pt}
    \setlength{\columnsep}{2pt}
    % --------------------------------------------------------------
    \section{1. Concepts}
    \textbf{Maturity} - The end of the life of a contract.

    \textbf{Arbitrage} - The simultaneous purchase and sale of the same asset in different markets in order to profit from tiny differences in the asset's listed price.

    \textbf{Market to Market (MtM)} - Price of an asset/product at the present moment.
    
    \textbf{Short} - Short position = Seller position
    
    \textbf{Long} - Long position = Buyer position
    
    \textbf{Risk factor} - Source of uncertainty.
    \begin{itemize}[label={--},leftmargin=4mm]
        \vspace{-1mm}
        \itemsep -.4mm
        \item IR: Interest Rate
            \subitem - Yield Curves: Rate curves given by countries
            \subitem - Reference rate curves: EURIBOR, LIBOR, SOFR
            \subitem - CDS curves
        \item FX: Foreign Exchange. ex. EUR/USD
        \item Equity: ex. BNP equity.
        \item Commodities: ex. Oil price.

    \end{itemize}

    \textbf{Spot Price} - The price for immediate delivery.
    
    \textbf{OTC Market} - Over-the-counter. A market where traders deal directly with each other or through an interdealer broker. The traders are usually financial institutions, corporations, and fund managers.
    
    \textbf{Exchange-traded markets} - A derivatives exchange is a market where individuals and companies trade standardized contracts that have been defined by the exchange. Contrary to OTC
    
    \section{2. Notation}
    \textbf{$S$} - Stock price, more generally underlying asset price.
    
    \textbf{$\Delta$} - Variation
    
    \textbf{$t$} - Time

    \textbf{$\phi$} - Normal distribution
    
    \textbf{$\sigma$} - Volatility

    \textbf{$r$} - Interest rate

    \textbf{$c$,$f$} - Price of option
    % ----------------------------------------------------------------
    \smallskip
    \smallskip
    \smallskip
    \smallskip

    % \columnbreak
    \section{3. Assets}
    Most known:
    \subsection{- \underline{Commodities}} Gold, Oil, Corn, etc.
    \subsection{- \underline{Real State}} Houses, buildings, lands, etc.
    \subsection{- \underline{Currencies}} EUR, USD, GBP, etc.
    \subsection{- \underline{Stocks}} BNP, Apple, etc.
    \subsection{- \underline{Bonds}}
    Types of bonds:
    Zero coupon bond, Coupon bond, Convertible bond, Callable bond
    \begin{itemize}[label={--},leftmargin=4mm]
        \vspace{-1mm}
        \itemsep -.4mm
        \item Zero coupon bond: Only has one face value payment at maturity.
        \item Coupon bond: It has multiple coupon payments and one face value payment.
            $\text{Coupon rate} = (k*n)/N, k:$ number of coupons per year.
        \item Convertible bond: Only convert to stock when market price of the bond $<$ value of the stock.
        \item Callable bond: Have to return the bonds for the company at a fixed price when the company decides to call it back.
    \end{itemize}

    \textbf{Pricing:}
    
    - Zero coupon bond:
    $$P=\frac{N}{(1+r_T)^T}$$
    - Coupon bond:
    
    
    $\Rightarrow$ Method 1:
    
    Continuously compounded and compounded once per annum:
    $$P=Ne^{-r_TT} + \sum_{i=i_0}^T Ce^{-ir_i}, P=\frac{N}{(1+r_T)^T} + \sum_{i=i_0}^T \frac{C}{(1+r_i)^i}$$

    $\Rightarrow$ Method 2 (Yield):
    
    Continuously compounded and compounded once per annum:
    $$P=Ne^{-yT} + \sum_{i=i_0}^T Ce^{-iy}, P=\frac{N}{(1+y)^T} + \sum_{i=i_0}^T \frac{C}{(1+y)^{i}}$$

    $P:$ Bond price
    
    $N:$ Principal / face value / value

    $C:$ Coupon

    $i_0:$ Time for receiving first coupon with years as unit. Ex: 1/2 year, 1 year, etc. 
    
    $i:$ Time. The difference between two consecutives $i$ is $i_0$
    
    $r_i:$ Rate interest at time $i$

    $T:$ Maturity

    $y:$ Yield
    % \columnbreak
    \section{4. Credit Risk}
    Probability of default by time $t$
    $$Q(t) = 1 - e^{-\int_{0}^{t}\lambda(\tau)d \tau} = 1 - e^{-\overline{\lambda}(t)t}, \qquad \overline{\lambda}(T) = \frac{s(T)}{1-R}$$
    $\lambda:$ Hazard rate

    $s(T):$ Bond yield spread for a T-year bond.
    
    $R:$ Recovery rate. Percentage of the bond value that is recovered in case of default.
    
    $$Q(t) = 1 - e^{-\frac{s(T)}{1-R}t}$$
    $$V(t) = e^{-\frac{s(T)}{1-R}t}$$

    $V(t):$ Probability of non-default by time $t$.

    \subsection{CVA and DVA}
    CVA: Credit Value Adjustment. 
    
    DVA: Debt Value Adjustment.
    $$CVA = \sum_{i=1}^{N}q_i v_i \qquad DVA = \sum_{i=1}^{N}q_i^* v_i^*$$
    $q_i:$ Risk-neutral probability of default of counterparty during the $i$th interval.
    
    $v_i:$ Present value of the expected loss to the bank if the counterparty defaults during the $i$th interval.
    
    $q_i^*:$ Risk-neutral probability of default of the bank during the $i$th interval.
    
    $v_i^*:$ Present value of the expected loss to the counterparty if the bank defaults during the $i$th interval.
    
    $q_i = V(t_{i-1}) - V(t_{i})$
    
    $v_i = f_{nd}(1-R)$

    \section{5. Derivatives}
    A derivative involves two parties agreeing to a future transaction. Its value depends
on (or derives from) the values of other underlying variables.
    
    \subsection{\underline{Forward contracts}}
    - A contract that obligates the holder to buy or sell an asset for a predetermined delivery price at a predetermined future time.
    The contract has a virtual zero cost at time zero.
    
    - \textbf{Forward's delivery price when created}

    $$F_0 = S_0e^{rT}$$

    $F_0:$ Forward delivery price
    
    $S_0:$ Price of underlying asset 
    
    $T:$ Time to maturity 
    
    $r:$ Risk-free rate 

    $\Rightarrow $\textbf{Assumptions}: 

    - The gain is the free-risk rate

    - Not coupon neither yield from underlying asset.  

    \vspace{2mm}
    - \textbf{MtM of a Forward} (Payoff)

    $$f = (F_0-K)e^{-rT}$$
    
    $f:$ MtM of Forward

    $F_0:$ Forward price at present time (if it was created at present time). Formula is above.
    
    $K:$ Delivery price for a contract that was negotiated some time ago. ($F_0$ when forward was created.)

    $T:$ Time to maturity 
    
    $r:$ Risk-free rate 

    
    \subsection{\underline{Future contracts}}
    - A contract that obligates the holder to buy or sell an asset at a predetermined delivery price during a specified future time period. The contract is settled daily.
    
    \subsection{Difference Forward and Future} - Future are traded in exchange markets whereas Forwards are OTC.

    \subsection{\underline{Options}} - 
    \subsection{\underline{Swaps: Interest rate swap}} - 
    \subsection{\underline{Swaps: Currency swap}} - 
    \subsection{\underline{Swaps: CDS(Credit default swap) }} - 
    \subsection{\underline{Swaps: Quanto }} - 
    % \columnbreak
    \section{6. Black and Scholles}
    \subsection{\underline{Assumptions}}
    \begin{itemize}[label={--},leftmargin=4mm]
        \vspace{-1mm}
        \itemsep -.4mm
        \item Stock price assumes that percentage changes in very short period of time are normally distributed.
        $$\frac{\Delta S}{S} \backsim \phi(\mu\Delta t, \Delta t)$$
    \end{itemize}
    \subsection{\underline{Equation}}
    $$\frac{\partial f}{\partial t} + rS\frac{\partial f}{\partial S} + \frac{1}{2} \sigma^2 S^2\frac{\partial ^2 f}{\partial S^2} = rf$$

    % \columnbreak
    \section{7. Risk}
    \subsection{\underline{Greeks or Risk sensitivities}}
    \begin{center}
        \footnotesize
        \begin{tabular}{ |c|c|c|c| }
            \hline
            Greek       & Symbol               & Measures  & Definition              \\
            \hline
            Delta & $\Delta = \frac{\partial c}{\partial S}$ & \thead{Underlying \\variable\\ (S) exposure} & \thead{Change in \\option price \\due to spot}\\
            Gamma & $\Gamma = \frac{\partial ^2 c}{\partial S^2}$ & \thead{Underlying \\variable\\ (S) convexity} & \thead{Curvature of \\option price \\with respect to spot}\\
            Theta & $\Theta = \frac{\partial c}{\partial t}$ & \thead{Time decay} & \thead{Change in \\option price \\due to time passing}\\
            Vega & $\upsilon = \frac{\partial c}{\partial \sigma}$ & \thead{Volatility \\ exposure} & \thead{Change in \\option price \\due to volatility}\\
            Rho & $\rho = \frac{\partial c}{\partial r}$ & \thead{Interest rate \\ exposure} & \thead{Change in \\option price \\due to interest rates}\\
            Volga & $\frac{\partial ^2 c}{\partial \sigma ^2}$ & \thead{Volatility \\ convexity} & \thead{Curvature of \\option price \\with respect to spot}\\
            Vanna & $\frac{\partial c}{\partial S \partial t }$ & \thead{} & \thead{Change in Delta \\due to Volatility}\\
            \hline
        \end{tabular}
    \end{center}

    \subsection{\underline{VaR \& ES}}

    \subsection{\underline{PnL}}

\end{multicols}

\end{document}